\chapter{Overview}

We are presenting the ontology which drives the data gathering and integration done as part of the project \emph{Enslaved: People of the Historic Slave Trade},\footnote{\url{http://enslaved.org/}} funded by The Andrew W. Mellon Foundation through Michigan State University's Matrix: The Center for digital Humanities \&{} Social Sciences.

Development of the ontology was a collaborative effort and was carried out using the principles laid out in, e.g., \cite{KrisnadhiHJHACC15,KrisnadhiH16,KrisnadhiKHARJ16}. The modeling team included domain experts, data experts, software developers, and ontology engineers. 

The ontology has, in particular, be developed as a \emph{modular} ontology \cite{HitzlerGJKP17} based on ontology design patterns \cite{HGJKP2016}. This means, in a nutshell, that we first identified key terms relating to the data content and expert perspectives on the domain to be modeled, and then developed ontology modules for these terms. The resulting modules, which were informed by corresponding ontology design patterns, are listed and discussed in Chapter \ref{sec:mods}. The Enslaved Ontology, assembled from these modules, is then presented in Chapter \ref{chap:ontology}.

For background regarding Semantic Web standards, in particular the Web Ontology Language OWL, including its relation to description logics, we refer the reader to \cite{owl2-primer, FOST}.

\input{primer.tex}
